\documentclass{article}
% Margin control
\usepackage[a4paper, total={7in, 11in}]{geometry}
% For Chinese support, recommended by Chou : augmented with xpinyin by sobenlee@gmail.com
\usepackage{CJKutf8}
% For xpinyin
\usepackage{xpinyin}
\usepackage[T1]{fontenc}
\usepackage{lmodern}
% From Chou
\usepackage{ucs} 
\usepackage[encapsulated]{CJK}
% Font choices
% bsmi limited - lacks certain characters
%\newcommand{\myfont}{bsmi} % or {stheiti}, etc
% gbsn recommended from sobenlee@gmail.com in xpinyin documentation
\newcommand{\myfont}{gbsn} % or {stheiti}, etc
% For Exam/Question support taken from en.wikibooks.org/wiki/LaTeX/List\_Structures
\usepackage{tasks}
\usepackage{exsheets}
\SetupExSheets[question]{type=exam}
% Personal macros
\newcommand{\cvoc}[1]{#1 & \xpinyin*{#1} }
\newcommand{\cvct}[3]{#1 & \xpinyin*{#1} & \pinyin{#2} & #3 \\ \hline}
\newcommand{\cvctp}[4]{#1 & \xpinyin*{#1} & \pinyin{#2} & #3 & #4 \\ \hline}
%\newcommand{\cvctp}[4]{#1 & \xpinyin*{#1} & \pinyin{#2} & #3 & #4 \\ \hline}
% xpe = xpinyin : english
\newcommand{\xpe}[2]{\xpinyin*{#1} : #2}

\begin{document} 
\begin{CJK}{UTF8}{\myfont} 
\section{Lesson 五}

Thursday 15/11/2018

\subsection{Homework Checks}

\subsubsection{Dictation Practice}

From week three, we should know (be able to read and write) the characters :

Dictation exercises as given, with original and then correct answers.

\begin{question}
  \pinyin{ni3 zhu4 zai4 na3 li}
  \xpinyin*{你住在哪里}
  \begin{tasks}(1)
  \task你住在哪里
  \end{tasks}
\end{question}

i.e. 5/5 correct.  Take care with the form of 你.

\begin{question}
  \pinyin{wo3 shi ying1 guo2 ren2}
  \xpinyin*{我是英国人}
  \begin{tasks}(1)
  \task我是英国人
  \end{tasks}
\end{question}

5.5/6 as my character for 是 had an inaccurate vertical bar across the 'tongue in the head'.

\begin{question}
  \pinyin{ta1 shi ming xing}
  \xpinyin*{他是明星}
  \begin{tasks}(1)
  \task他失明星
  \end{tasks}
\end{question}

5.5/6 as again my character for 星 again had a spurious extra vertical bar through the 'tongue in the head'.

Overall score for dictation : 16 /17 (94.1\%).

\section{Character Review}

I now know approximately 50 characters from the main sheets and others picked up.

\subsection{Unit 3 : What do you do? }

What do you do?  What is your occupation?  Some cultural/linguistic background.  The word for work, 工作, can be read as a noun (the type of work) or a verb (to work).  A strongly related verb is 做 (to do).  When discussing place of work, {\it place} comes prior to the verb (and this applies in general to all verbs).   e.g. 我在牛津工作 (I in Oxford (nui1jin1) work).  The word yuan4 院 indicates a place of work.  Thus we can form the words for school (to study, place) 学院  or hospital (medical place) 医院 in a logical fashion.   Similarly the word sheng1 生 indicates a person and so we can form student (to study, person) as 学生 or doctor (medical, person) 医生.

Note that the tone of sheng varies according to the tone of prefixes.   Thus prefixed by xia4 or xue2 the sheng is toneless, but prefixed by yi1 remains at tone1.

Useful vocabulary :

    \begin{tabular}{|l|l|l|l|l|} \hline
      \cvctp{那}{na3}{interrogative}{which}
      \cvctp{作}{zuo4}{verb}{to do}
      \cvctp{都}{dou1}{adverb}{all/both}
      \cvctp{记者}{jizhe3}{noun}{journalist (jeezhoua)}
      \cvctp{老师}{lao3shi1}{noun}{old+master=teacher}
      \cvctp{家}{jia1}{noun}{family/home}
      \cvctp{赵}{zhao4}{verb}{to take}
      \cvctp{有}{you3}{verb}{to have}
      \cvctp{}{}{}{}
      \cvctp{}{}{}{}
      
    \end{tabular}

    Phrases in passing

    \begin{enumerate}
    \item \xpe{你作什么工作?}{You do what work?}
    \item \xpe{我在牛津工作}{I in Oxford work. (note place before verb)}
    \item \xpe{我们都是学生}{We all are students.}
    \item \xpe{你家即口人}{Your family how many (total person measure word) people?}
    \item \xpe{我有三口人}{My family has 3 (total person measure word) people.}
    \item \xpe{我有两个积极}{I have two (measure word general) elder sisters.}
    \item \xpe{我妹妹是学生。我姐姐是老师}{My younger sister is a student.  My elder sister is a teacher.}
    \item \xpe{}{xiong di = all brothers (to do look up characters)}
    \item \xpe{}{jia mei = all sisters (to do look up characters) }
    \item \xpe{}{}
    \end{enumerate}


      
      \subsection{Grammar, Adverbs}

      Recall - subject, adverb, verb, object.  Adverbs come before the verb always.   The order will in general be whatever is correct to imply the intended meaning.  e.g. If combining 不, 都 then we can have either 不都 meaning not all, or 都不 meaning all not, i.e. none.   THe adverb 也 is special and almost always precedes all other adverbs.

      The structural particle 的 is used to indicate the posessive and can thus be used as an apostrophical suffix to replcae similar usage in English.   Thus Adam 的书 is Adam's book.  For entities with close relationships it is permissible to omit the particle which could otherwise yield unwieldy repetition.   Consider brother's uncle's home's location which would be 弟弟的爸爸的家的元.  Note all the ``de''s.   So one can talk about 我的书 but simply 我妹妹.

      Measure words and why two has two forms.   When referring to a pair of younger sisters, one would not use the ordinal/counting form of the second positive integer 二 but instead the variant 两.   When referring to numbers of nouns a linking {\it measure word} is required between the number and the noun.  The most generic (best guess) is 个 and so one would refer to 两个弟弟 or 三个妹妹 but there is a specific measure word for all the mouths in the family (i.e. the total number of family members) which is 口.

      The conjunction and (和)is not used to join two sentences.   Just split them.


\subsection{New Words}

Taken from the vocabulary in the text book, unit 1, supplemented with some of the other words introduced in the lesson.

    \begin{tabular}{|l|l|l|l|l|} \hline
      \cvctp{您}{nin}{adv}{You (polite)} 
    \end{tabular}

    \subsection{Culture and Usage}
\subsection{Character Study}

A new dictation sheet (DC1 Unit1) is provided with the followin gcharacters to learn.

\subsection{Homework}

\subsubsection{Dictation}

DC1 Unit 2 vocabulary.

\subsubsection{Oral}

Prepare to introduce your family to others.

\subsubsection{Written Work}

To be handed in and checked.  New vocab and grammar sheets to be completed.

\subsubsection{Class Times/Rooms}

Tuesday 9-11 in 301.  Thursday 5-7 in 301.

\subsection{IP/Copyright}

In respect of the content, Dr. Wendy Che, and to an extent yet to be determined, her employer, the University of Oxford Language Center.

In respect of the write up and typesetting, Dr. Adam Vercingetorix Stephen.

\end{CJK} 
\end{document}
