\documentclass{article}
% Margin control
\usepackage[a4paper, total={7in, 11in}]{geometry}
% For Chinese support, recommended by Chou : augmented with xpinyin by sobenlee@gmail.com
\usepackage{CJKutf8}
% For xpinyin
\usepackage{xpinyin}
\usepackage[T1]{fontenc}
\usepackage{lmodern}
% From Chou
\usepackage{ucs} 
\usepackage[encapsulated]{CJK}
% Font choices
% bsmi limited - lacks certain characters
%\newcommand{\myfont}{bsmi} % or {stheiti}, etc
% gbsn recommended from sobenlee@gmail.com in xpinyin documentation
\newcommand{\myfont}{gbsn} % or {stheiti}, etc
% For Exam/Question support taken from en.wikibooks.org/wiki/LaTeX/List\_Structures
\usepackage{tasks}
\usepackage{exsheets}
\SetupExSheets[question]{type=exam}
\newcommand{\cvclt}[4]{#1 & \xpinyin*{#1} & \pinyin{#2} & #3 & #4 \\ \hline}
\newcommand{\vclt}[4]{\xpinyin*{#1} & \pinyin{#2} & #3 & #4 \\ \hline}
\begin{document} 
\begin{CJK}{UTF8}{\myfont} 
\section{Lesson 二}

Wednesday 24/10/2018

\subsection{Culture}

Since the open doors policy of the 80s/90s and the acceptance of large numbers of western visitors to China, discourse is becoming more polite.  Historically, it would have been normal to generally ignore people outwith one's close social circle.  Thus it would have been seen as odd to greet a stranger in the street, but this is now becoming more acceptable.

Be careful in some translations to be aware of subtle implications.   For example, \xpinyin*{请} (invite; please) can imply when inviting friends or colleagues that the invitation is all-inclusive and at the generosity of the inviting host.   Thus to invite a group to dinner can be an expensive business.

b\subsection{New Words}


\subsection{Question Styles and Word Order}

The word \xpinyin*{也} (also)\footnote{Note that the character \xpinyin*{也} also forms part of the word 他 with the addition of the person or man radical, hence he.} is used at the beginning of sentences.   Thus, 'too student I am' in place of 'I am also a student' or 'I am a student too' in English.

Questions can be asked in a number of ways, using question particles (ne, ma), question words (how, what, when, why, where), using an implicit style (statement, opposite) or by convention as an interpretation (you good).  Note again the advice from lesson one.

Word order in general is subject adverb adjective where the (adverb adjective) form a predicate pair and in many statements no verb is required.  This is in contrast to typical western phrasing of subject verb object where the (verb object) form the predicate.   Thus : 'you are well' can become simply 'you well?' but usually the Chinese will throw in an adjective such as '好(very)' not as a superlative but just bolstering the (adverb adjective) predicate.  Thus more typically, 我很好 in place of 我好.  In the negative, also such as 我不好.

  Note that it would be considered foreign and unusual to actually use the classic greeing (你好马)with strangers since there is an implied relationship yet to be established before one would enquire as to the well being of another.

\subsection{Numbers}

The numbers 0--10 were introduced as below, as well as a single hand based counting system.   For radio code, a variant of the number one which are phonetically distinct from the number seven (otherwise phonetically similar) is sometimes used for clarity.  As an exercise, the class exchanged phone numbers to practice.  Each numeral has a character, but roman numeric notation is often used, especially for longer numbers.

  \begin{tabular}{|l|l|l|l|l|} \hline
    \cvclt{零}{ling2}{Bunched fist}{Zero}
    \cvclt{一}{yi1}{Index finger}{One}
    \cvclt{二}{er4}{Index plus middle fingers V}{Two}
    \cvclt{三}{san1}{Thumb and index make circle, remaining three fingers up.}{Three}
    \cvclt{四}{si4}{All fingers}{Four}
    \cvclt{五}{wu3}{All fingers plus thumb}{Five}
    \cvclt{六}{liu4}{Thumb plus pinkie}{Six}
    \cvclt{七}{qi1}{Thumb touching index plus middle}{Seven}
    \cvclt{八}{ba1}{Thumb and index form an L}{Eight}
    \cvclt{九}{jiu3}{Hook index finger}{Nine}
    \cvclt{十}{shi}{As zero, or cross both index fingers as per the character}{Ten}
    
  \end{tabular}


  \subsection{Chinese Characters}

  Historically, pictographs with a nominal derivation from a small sketch.  e.g. 山 looks like a mountain.   Formed from a range of standard strokes and dots which are grouped into component parts with semantics forming full characters (words) then joined into multiple characters to give phrases.

  Stroke order, with a few exceptions

  \begin{enumerate}
  \item Top to bottom
  \item Left to right
  \item Horizontal before vertical, except bottom lines closing
  \item Frame before contents
  \item Middle before sides
  \item Secondary dots last
  \end{enumerate}

  Chinese school exam questions would expect a candidate to be able to answer: 'which is the 4th stroke when forming the character 五'?

  Learning : always use the same stroke order.   Find mental imagery to help understand the characters.  Recognise the key radical components (214).

  Compound characters build meaning.  Note that in combination, the parts scale so that all characters always occupy one box of space.  e.g. By combining 日 (ri2, sun) and 月 (yue4, moon) we get 明 (ming2, bright).   Other combinations, noting that the logically combined characters often get aggregated into a new form include

      \begin{tabular}{|l|l|l|l|l|} \hline
        \cvclt{木木 林}{mu4mu4 lin2}{Radical for tree repeated twice.}{forest, grove}
        \cvclt{人木 休}{ren2mu4 xiu4}{Person + wood = lean on tree}{rest.}
        \cvclt{女字 好}{nv3zi3 hao3}{Female + son = good life}{good, well}
        \cvclt{人人 从}{ren2ren2 cong2}{Person + person = follow}{follow}
        \cvclt{人人人 众}{ren2ren2ren2 zhong4}{person+person+person = crowd}{crowd}
      \end{tabular}

      Narratives id the memory.

      It is critical to learn the characters.  Even the 56 stroke character Biang!  The 214 base readicals help to understand, and are a key in some dictionaries.
      
\subsection{Example Phrases}

To be fully transcribed as an exercise :

\begin{tabular}{|l|l|l|l|} \hline
  \vclt{你}{ni3}{you}{you}
  \vclt{对不起我来晚了}{duibuqi wo3 lai2 wan3 le}{Excuse me I come late past-particle.}{Sorry I am late.}
  \vclt{请坐}{qing3 zuo4}{Please sit.}{Please sit.}
  \vclt{你怎么样}{ni3 zen3 me yang2}{You how are things?}{How are you?}
  \vclt{你呢}{ni3 ne}{You repeat-question-particle}{And you?}
  \vclt{你忙不忙}{ni3 mang2 bu4 mang2}{You busy not busy.}{Are you busy?}
  \vclt{不忙你忙}{bu4 mang2 ni3 mang2 }{You busy not busy.}{You're busy? (really/rhetorical}
  \vclt{你好}{ni3 hao3 ma}{You good question-particle}{Are you well?}
\end{tabular}


\subsection{Homework}

Learn the characters for 1-10 plus basic greetings and pronouns to be tested in dictation (see also the week by week dictation PDF from these writeups).

Oral homework : be ready to use the greetings II vocabulary to have basic sentences in class.

  \subsection{IP/Copyright}
  
  In respect of the content, Dr. Wendy Che, and to an extent yet to be determined, her employer, the University of Oxford Language Center.

In respect of the write up and typesetting, Dr. Adam Vercingetorix Stephen.


\end{CJK} 
\end{document}
