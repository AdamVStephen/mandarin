\documentclass{article}
% Margin control
\usepackage[a4paper, total={7in, 11in}]{geometry}
% For Chinese support, recommended by Chou : augmented with xpinyin by sobenlee@gmail.com
\usepackage{CJKutf8}
% For xpinyin
\usepackage{xpinyin}
\usepackage[T1]{fontenc}
\usepackage{lmodern}
% From Chou
\usepackage{ucs} 
\usepackage[encapsulated]{CJK}
% Font choices
% bsmi limited - lacks certain characters
%\newcommand{\myfont}{bsmi} % or {stheiti}, etc
% gbsn recommended from sobenlee@gmail.com in xpinyin documentation
\newcommand{\myfont}{gbsn} % or {stheiti}, etc
% For Exam/Question support taken from en.wikibooks.org/wiki/LaTeX/List\_Structures
\usepackage{tasks}
\usepackage{exsheets}
\SetupExSheets[question]{type=exam}
\begin{document} 
\begin{CJK}{UTF8}{\myfont} 
\section{Lesson 一}

Wednesday 17/10/2018.

\subsection{Course Admin}

Introduction week.   The course will be based around the {\it Discover China} textbook (ISBN 9780230405950) and optionally the accompanying workbook (ISBN 978923946384).  Use of a hard copy dictionary as a good means of memory training is recommended and the Oxford Starter Chinese Dictionary is a good choice.

Our course tutor is Dr Wenbing (Wendy) Che who has an impressive CV and set of qualifications including batchelor's degree in Mechanical Engineering, Doctorate in Business Administration, Diploma in Chinese Medicine and a long and successful track record of teaching Mandarin to business and academic students.    The course runs from W2-W7 in Michaelmas Term (6 classes) and from W1-W7 in Hilary and Trinity Terms.  Certificates of attendance (>80\%) and competence (by exam on reading/writing/listening) may be awarded.

The course contents for Michaelmas Term will be :

\begin{enumerate}
\item W2 Introduction
\item W3 Greetings and Numbers
\item W4 Discover China Unit 1
\item W5 Discover China Unit 2
\item W6 Discover China Unit 3
\item W7 Discover China Unit 4
\end{enumerate}

\subsection{Mandarin Background}

Mandarin Chinese is the official language of China and the primary dialect of Chinese.  There are 10 major dialects, such as Cantonese.  Dialects are sufficiently different from one another that speakers may find it easier to communicate via a separate common language such as English.   Chinese has fewer phonetic combinations than many Western languages and augments the set of sounds using tones.   Mandarin has foru tones (flat, rising, falling/rising, falling) plus the neutral tone.   Cantonese has eight.   The written language is character based.  Romanised phonetic notation, called pinyin, is used to help beginners learn the sounds, but it is important to learn the characters at the same time.  Pinyin notation includes stress or tone marks.   Tones vary where syllables accrete, though the notation is not changed.

\subsection{Pinyin}

The 26 roman letters are used to indicate 23 consonants/initials, 35 vowels/finals.  Hence there are theoretically $$23\times35=805$$ syllable combinations, though not all are used in practice.   Youtube videos showing how to physically move mouth and tongue to achieve the proper sounds are recommanded.   Tones are critical to comprehensibility.  e.g. The syllable 'Ma' has five different meanings : \xpinyin*{马玛吗麻妈}.

Using the xpinyin package

\begin{tasks}(5)
  \task \xpinyin*{马}
  \task \xpinyin*{玛}
  \task \xpinyin*{吗}
  \task \xpinyin*{麻}
  \task \xpinyin*{妈}
\end{tasks}

\subsection{Written Chinese}

A typical, educated vocabulary will be of the order of 8,000 characters.  A newspaper will use around 2,000.  A tourist sign vocabulary would run to 800.  GCSE teaches around 500.  This course and the text book will develop around 300.

A proposal to move strictly to a phonetic system was made historically, and neatly refuted bu the poet Zhao Yanren via his.

The written form comprises :

\begin{enumerate}
\item Strokes
\item Characters
\item Words
\item Phrases
\end{enumerate}

Two written forms : Simplified and Traditional, exist for aroudn 1000 characters.  Traditional is used exclusively in Hong Kong and Taiwan.  Simplified for the subset is used in mainland China and was introduced to increase literacy rapidly, particularly among women.

\subsection{Grammar}

A few key points.   Some are differences from English.  Some are considerable simplifications compared to other languages.

\begin{enumerate}
\item Word order differs.
\item Time is money, and hence temporal qualifiers come first.
\item Interrogative is distinguished from declarative by throwing in a question particle.  e.g. You are students (statement).  You are students ma? (question, introduced with the ma particle).
\item Verbs have no tense : aspects of action (impending/progressive/continuous/conclusive/historical) are indicated via use of aspect particles or model particles.
\item Verbs have no personal conjugation.  Pronoun plus verb are used simply.
\item Prepositions are not required.
\item Optative verbs have a function (TODO: look up what this means).
\end{enumerate}

A few differences, one of which adds a little more complexity:

\begin{enumerate}
\item Measure words are required between numbers and nouns.  This occurs too in English : e.g. a cup of coffee (measure word: cup).   In Chinese there are around 300 measure words. They have a rather elaborate sophistication, though younger speakers are tending to reduce the usage where they can. e.g. Rather than separate measure words for brother/sister/sibling, settle on one common one for this group.  If in doubt, the most comment measure word is \xpinyin*{ge} - so this is a good guess.
\item Always put largest entities/concepts first.  So dates begin with the year.  Identity starts from the nation and builds down.  Culturally, the society is more important than the individual, because logically, the individual has a dependency on the pre-existence of the family/society within which they find themselves.
\end{enumerate}

\subsection{Practical Speaking Exercise}

The group followed pinyin sound exercises based on AV media, being introduced to ``bopomofo'' and so forth.

\subsection{Homework}

For lesson two :

\begin{enumerate}
\item Learn greetings from the handout sheet.
\item Practice pinyin sounds.
\end{enumerate}

\subsection{Online Resources}

\begin{enumerate}
\item Oxford University Weblearn account where class resources will be uploaded.
\item mdbg.net--Online dictionary with multiple useful featuers including quiz/translation/fully featured listings.
\item yellowbridge.com
\item archchinese.com
\item pieco.com app
\item Beijin University course for online characters for beginners (coursera).
\item www.livethelanguage.cn/how-to-learn-chinese/
\end{enumerate}

\subsection{Handout Vocabulary}

From the class handouts this week, to learn for next week :


\subsection{Additional Vocabulary}

Also psotted in passing

\begin{enumerate}
\item Person : \xpinyin*{人}
\item Dragon : \xpinyin*{}
\item To buy :\xpinyin*{}
\item To speak : \xpinyin*{}
\item Country : \xpinyin*{国}
\item Flower: \xpinyin*{hua}
\end{enumerate}

\subsection{Phrases}
I picked up the following phrases in passing during the class :

\subsection{Quiz}

\begin{tasks}(5)
  \task 一
  \task 二
  \task 三
  \task 四
  \task 五
\end{tasks}

\begin{question}
  What is the character for the numeral 4 ?
  \begin{tasks}(5)
  \task 一
  \task 二
  \task 三
  \task 四
  \task 五
\end{tasks}

\end{question}


\subsection{How This Was Made}

\LaTeX setup notes as follows :

\begin{enumerate}
\item Set up CJK etc using Ubuntu packages and notes from Dr Chou.
\item List structures and exam style questions references via en.wikibooks.org/wiki/LaTeX/List\_Structures
\end{enumerate}

\subsection{IP/Copyright}

In respect of the content, Dr. Wendy Che, and to an extent yet to be determined, her employer, the University of Oxford Language Center.

In respect of the write up and typesetting, Dr. Adam Vercingetorix Stephen.


\end{CJK} 
\end{document}
