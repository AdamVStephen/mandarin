\documentclass{article}
% For Chinese support, recommended by Chou : augmented with xpinyin by sobenlee@gmail.com
\usepackage{CJKutf8}
% For xpinyin
\usepackage{xpinyin}
\usepackage[T1]{fontenc}
\usepackage{lmodern}
% From Chou
\usepackage{ucs} 
\usepackage[encapsulated]{CJK}
% Font choices
% bsmi limited - lacks certain characters
%\newcommand{\myfont}{bsmi} % or {stheiti}, etc
% gbsn recommended from sobenlee@gmail.com in xpinyin documentation
\newcommand{\myfont}{gbsn} % or {stheiti}, etc
% For Exam/Question support taken from en.wikibooks.org/wiki/LaTeX/List\_Structures
\usepackage{tasks}
\usepackage{exsheets}
\SetupExSheets[question]{type=exam}
% Personal macros
\newcommand{\cvoc}[1]{#1 & \xpinyin*{#1} }
\newcommand{\cvct}[3]{#1 & \xpinyin*{#1} & \pinyin{#2} & #3 \\ \hline}
\newcommand{\cvctp}[4]{#1 & \xpinyin*{#1} & #2 & \pinyin{#3} & #4 \\ \hline}
\begin{document} 
\begin{CJK}{UTF8}{\myfont} 
  
  \section{Vocabulary}
  \Large

  \subsection{Greeting Phrases}

  From lesson one.
  
  \begin{tabular}{|l|l|l|l|} \hline
    % hanzi & \xpinyin*{hanzi} & \pinyin{code} & translation \\ 
    你 & \xpinyin*{你}  & \pinyin{ni3}  & you \\ \hline
    好 & \xpinyin*{好}  & \pinyin{hao3}  & \\ \hline

    \cvoc{好} & \pinyin{hao3} & good, fine; well \\ \hline

    \cvct{好}{hao3}{good, fine; well. Via macro}
    \cvct{上午好}{shang4wu3 hao3}{Good morning}
    \cvct{下午好}{xia4wu3hao3}{Good afternoon}
    \cvct{晚上好}{wan3shang hao3}{Good evening}
    \cvct{谢谢}{xie4xie}{Thanks (this character is a nightmare to find)}
    \cvct{再见}{zai4jian4}{Goodbye}

  \end{tabular}
%  \vfill\eject

  \subsection{Personal Pronouns}

  From lesson one.
  
  \begin{tabular}{|l|l|l|l|} \hline
    \cvct{我}{wo3}{I, me}
    \cvct{你}{ni3}{you}
    \cvct{他}{ta1}{he}
    \cvct{她}{ta1}{she}
    \cvct{它}{ta1}{it}

    \cvct{我们}{wo3men}{we}
    \cvct{你们}{ni3men}{you}
    \cvct{他们}{ta1men}{they (masc)}
    \cvct{她们}{ta1men}{they (fem)}
    \cvct{它们}{ta1men}{they (neut)}
  \end{tabular}
  \vfill\eject

    \subsection{Numbers}

  From lesson two.

    \begin{tabular}{|l|l|l|l|} \hline
      \cvct{零}{ling2}{Zero}
      \cvct{一}{yi1}{One}
      \cvct{二}{er4}{Two}
      \cvct{三}{san1}{Three}
      \cvct{四}{si4}{Four}
      \cvct{五}{wu3}{Five}
      \cvct{六}{liu4}{Six}
      \cvct{七}{qi1}{Seven}
      \cvct{八}{ba1}{Eight}
      \cvct{九}{jiu3}{Nine}
      \cvct{十}{shi}{Ten}
      
    \end{tabular}

    \subsection{Greetings 2}

  From lesson two.

  Handy greetings.

    \begin{tabular}{|l|l|l|l|l|} \hline
      \cvctp{您}{nin}{adv}{You (polite)}
      \cvctp{不}{bu4}{adv}{no, not; negative prefix}
      \cvctp{不错}{bu4cuo2}{phr}{not bad/pretty good}
      \cvctp{还可以}{hai2ke3yi3}{phr}{OK, pretty good}
      \cvctp{马马虎虎}{ma3ma3hu1hu1}{phr}{so-so}
      \cvctp{怎么痒}{zen3meyang4}{phr}{How are things?}
      \cvctp{也}{ye3}{adv}{also, too, means things are the same}
    \end{tabular}

    Times of day

    \begin{tabular}{|l|l|l|l|l|} \hline
      \cvctp{早上}{zao3shang}{n}{Early morning (before 9am/work)}
      \cvctp{上午}{shang4wu3}{n}{Morning (9-lunch)}
      \cvctp{中午}{zhong1wu3}{n}{Noon (11.30-2)}
      \cvctp{下午}{xia4wu3}{n}{Afternoon (2-6/dark)}
      \cvctp{晚上}{wan3shang}{n}{Evening (6/dark onwards)}
    \end{tabular}
    
    (TODO) ... Noting some key subcomponents from the above.

    Some useful verbs

    \begin{tabular}{|l|l|l|l|l|} \hline
      \cvctp{来}{lai2}{v}{Come.}
      \cvctp{请}{qing1}{v}{Invite (and pay for).  Also please.}
      \cvctp{坐}{zuo4}{v}{Sit.}
      \cvctp{}{}{}{}
    \end{tabular}
    
    Some handy adverbs and adjectives
    
    \begin{tabular}{|l|l|l|l|l|} \hline
      \cvctp{忙}{mang2}{adj}{busy}
      \cvctp{累}{lei4}{adj}{tired(difficult character to locate in typing)}
      \cvctp{晚}{wan3}{adj}{late}
    \end{tabular}

    Two new particles
    
    \begin{tabular}{|l|l|l|l|l|} \hline
      \cvctp{了}{le}{par}{Past/Completed.  Forms the past tense with a verb.}
      \cvctp{呢}{ne}{par}{What about?  And ?  Where is?  Highly useful and flexible particle.}
    \end{tabular}

    Finally - a few very handy phrases
    
    \begin{tabular}{|l|l|l|l|l|} \hline
      \cvctp{不客气}{bu2ke2qi}{}{}
      \cvctp{不用谢}{bu2yong4xie4}{}{}
      \cvctp{对不起}{dui4buqi3}{}{}
      \cvctp{没}{mei2guan1xi}{}{}
    \end{tabular}

    \section{Dialogues}

    \subsection{Lesson Two}

    
\subsection{How This Was Made}

\LaTeX setup notes as follows :

\begin{enumerate}
\item Set up CJK etc using Ubuntu packages and notes from Dr Chou.
\item List structures and exam style questions references via en.wikibooks.org/wiki/LaTeX/List\_Structures
\end{enumerate}

\subsection{IP/Copyright}

In respect of the content, Dr. Wendy Che, and to an extent yet to be determined, her employer, the University of Oxford Language Center.

In respect of the write up and typesetting, Dr. Adam Vercingetorix Stephen.


\end{CJK} 
\end{document}
