\documentclass{article}
% Margin control
\usepackage[a4paper, total={7in, 11in}]{geometry}
% For Chinese support, recommended by Chou : augmented with xpinyin by sobenlee@gmail.com
\usepackage{CJKutf8}
% For xpinyin
\usepackage{xpinyin}
\usepackage[T1]{fontenc}
\usepackage{lmodern}
% From Chou
\usepackage{ucs} 
\usepackage[encapsulated]{CJK}
% Font choices
% bsmi limited - lacks certain characters
%\newcommand{\myfont}{bsmi} % or {stheiti}, etc
% gbsn recommended from sobenlee@gmail.com in xpinyin documentation
\newcommand{\myfont}{gbsn} % or {stheiti}, etc
% For Exam/Question support taken from en.wikibooks.org/wiki/LaTeX/List\_Structures
\usepackage{tasks}
\usepackage{exsheets}
\SetupExSheets[question]{type=exam}
% Personal macros
\newcommand{\cvoc}[1]{#1 & \xpinyin*{#1} }
\newcommand{\cvct}[3]{#1 & \xpinyin*{#1} & \pinyin{#2} & #3 \\ \hline}
\newcommand{\cvctp}[4]{#1 & \xpinyin*{#1} & \pinyin{#2} & #3 & #4 \\ \hline}
%\newcommand{\cvctp}[4]{#1 & \xpinyin*{#1} & \pinyin{#2} & #3 & #4 \\ \hline}
% xpe = xpinyin : english
\newcommand{\xpe}[2]{\xpinyin*{#1} : #2}

\begin{document} 
\begin{CJK}{UTF8}{\myfont} 
\section{Lesson 四}

Thursday 8/11/2018

\subsection{Homework Checks}

\subsubsection{Dictation Practice}

From week three, we should know (be able to read and write) the characters :

Dictation exercises as given, with original and then correct answers.

\begin{question}
  \pinyin{ta1 jiao4 shenme mingzi}
  \begin{tasks}(2)
  \task他X什么名字
  \task他叫什么名字 
  \end{tasks}
\end{question}

i.e. 5/6 correct, but did not know character for \xpinyin*{叫}.

\begin{question}
  \pinyin{wo3hen3gao1xing renshi ni3}
  \begin{tasks}(2)
  \task我X高兴认识你
  \task我很高兴认识你
  \end{tasks}
\end{question}

i.e. 6.5/8 but did not know character for \xpinyin*{很} and miswrote character for \xpinyin*{你}.

\begin{question}
  \pinyin{qing3wen4ni3xingwangma}
  \begin{tasks}(2)
  \task起X问你XXX
  \task 请问你王马
  \end{tasks}
\end{question}

i.e.3/6 but did not know character for \xpinyin*{请}, \xpinyin{王} or \xpinyin*{马}.

Overall score for dictation : 14.5 /20 (72.5\%).

\section{New Characters}

Characters tested in dictation but not previously provided on the dictation sheets :
    
    \begin{tabular}{|l|l|l|l|l|} \hline
      \cvctp{很}{hen3}{adj}{very}
      \cvctp{姓}{xing}{noun/verb}{family name}
      \cvctp{叫}{jiao4}{verb}{to be called}
      \cvctp{马}{ma}{particle}{question particle}
    \end{tabular}

In addition, from week two, we now should know (be able to read and write) the characters :

一, 二, 三, 四, 五, 六, 七, 八, 九, 十  (i.e. the numerals 1--10) as well as

你 好 ; 国 ; 人 ; 他 ; 我 们 

\subsection{Unit 2 : Where do you come from?}

First - some cultural explanation.  Most place names are transliterated using chinese phonemes.   One or two key UK exceptions are Oxford and Cambridge which are written semantically.   Some names combine both semantic and phonetic clues, such as South Africa (south as adjective) and New Zealand (new as adjective).   When disambiguating a city that exists in many countries, the country qualifer leads the city name using the usual principle of largest entity comes first.   To denote citizen of a country, add the person suffix.  To denote the country itself, add the country/kingdom suffix.    One can ask where someone comes from, or which country one comes from, or indeed which country one lives in.

In answering questions, note that it is generally good form to answer by repeating the assertion or negation statement, possibly with bu or shi as reinforcement.   Caveat, the verb 是 may only be negated by 不.

Useful vocabulary :

    \begin{tabular}{|l|l|l|l|l|} \hline
      \cvctp{那}{na3}{interrogative}{which}
      \cvctp{那里}{na4li}{interrogative}{where (in Beijing nar, In Shanghai, nali.}
      \cvctp{主在}{zhu4zai4}{verb}{to live (in)}
      \cvctp{牛津}{niu2jin1}{place}{Ox-Ford}
      \cvctp{洊桥}{jian4qiao4}{place}{Cambridge (first character to be checked/changed}
      \cvctp{明星}{ming2xing1}{noun}{bright-star (literally) -> celebrity}
    \end{tabular}

    Phrases built up (and reiterated advice to form questions by working back from the answer and substituting a question word for the answer noun).

    \begin{enumerate}
    \item \xpe{你是哪里人}{you are where person?}
    \item \xpe{你是哪国人}{you are what country person?}
    \item \xpe{他也是中国人}{he also is chinese person.}
    \item \xpe{他叫什么名字}{he is called what name?}
    \item \xpe{他是那国人}{he is which country person?}
    \item \xpe{他书在那里}{he lives in where?}
    \item \xpe{}{}
    \item \xpe{}{}
    \item \xpe{}{}
\end{enumerate}


    \section{Calligraphy and Characters}

    Introducing radicals 口 \pinyin{kou3} and 日 \pinyin{ri4}.

    Things involving mouth (口) will involve this radical.  For example question particles 马, 呢, also 叫 (to be called) and 吃 (to eat).

    Things involving sun include 明 (bright), 早 (early) and 是 (to be).

      To fit the space, or for other minor effects, radicals form may be transformed but should still be recognisable as a derivation.

      \section{Homework}

      \begin{enumerate}
      \item Characters : 20 new characters.
      \item Oral : prepare for more introductory discussions, about where you live/where you are from.
      \item Quiz 2 for unit 2.
      \end{enumerate}

      \section{Unit 3}

      New vocabulary from p.39 of the textbook.
      
\subsection{Number 11-99}

Very logical.  For multiples of ten, say the multiple, then ten.  Hence forty is ``four ten'' 四十.  Then add the units.  Hence
twenty-five is ``two-ten five'' or 二十五.

Practice

\begin{tabular}{|l|l|l|l|} \hline
    72 & 七十二 & 40 & 四十  \\ \hline
    23 & 二十三 & 51 & 五十一 \\ \hline
    14 & 十四   & 65 & 六十五 \\ \hline
    54 & 五十四 & 60 & 六十 \\ \hline
    76 & 七十六 & 18 & 十八 \\ \hline
    99 & 九十九 & 21 & 二十八 \\ \hline
    37 & 三十七 & 83 & 八十三 \\ \hline
\end{tabular}

\subsection{New Words}

Taken from the vocabulary in the text book, unit 1, supplemented with some of the other words introduced in the lesson.

    \begin{tabular}{|l|l|l|l|l|} \hline
      \cvctp{您}{nin}{adv}{You (polite)} 
      \cvctp{姓}{xing}{verb/noun}{Family name/Surname}
      \cvctp{名字}{mingzi}{verb/noun}{Name}
      \cvctp{问}{wen4}{verb}{ask}
      \cvctp{认识}{renshi}{verb}{to know}
      \cvctp{叫}{jiao4}{verb}{be called}
      \cvctp{呢}{ne}{particle}{Question particle}
      \cvctp{什么}{shen2me}{interrogative}{what}
      \cvctp{高兴}{gao1xing}{adj}{happy; glad}
      \cvctp{小姐}{xiao3jie3}{title}{Miss (young women)}
      \cvctp{打}{da3}{verb}{open}
      \cvctp{开}{kai1}{possessive}{your}
      \cvctp{书}{shu1}{noun}{book}
      \cvctp{页}{ye4}{noun}{page}
      \cvctp{学}{xue2}{verb}{learn}
      \cvctp{生伺}{sheng1ci}{noun}{new word/vocabulary}
      \cvctp{跟}{gen1}{verb}{follow}
      \cvctp{读}{du2}{verb}{to read/pronounce/study}
    \end{tabular}

    \subsection{Culture and Usage}

    Word order. Note that number comes before noun in the case of : 'open your books to 19 page'.

    Names can cause confusion due to order.    Order is greater meaning first.   Only use first names with close relationships.   Colleagues and classmates use full name, but never use both names for someone of senior rank - this would cause offence.  Use title, or title and surname.  Beware business card double inversion.

    \begin{tabular}{|l|l|}\hline
      UK & 中国 \\ \hline
      Given + Family Name & Family + Given Name \\ \hline
      Mr/Mrs/Miss + Family Name &  Family name + \xpinyin*{先生} / \xpinyin*{女士}/\xpinyin*{小姐} \\ \hline
      Minister + Family Name & Family + \xpinyin*{先生} \\ \hline
      Prof + Family Name & Family +  \xpinyin*{叫首} \\ \hline
      Dr + Family Name & Family +  \xpinyin*{伯是} \\ \hline
      Doctor (med) + Family Name & Family +  \xpinyin*{一生} \\ \hline
      
      Mrs + Husband Family Name &  Husband Family Name + tai4tai \\ \hline
    \end{tabular}

    TODO : correct the entries for Minister/Prof/Dr/Doctor?mrs based on the pinyin of bu4zhang3 ; jiao4shou4 ; bo2shi4; yi1sheng ; tai4tai

    Note that Chinese women retain their surname on marriage, so to form the Mrs Foo it is necessary to know the surname of their husband.   Conversely if you know Anna Wang, do not assume that her husband is Mr Wang.

\subsection{Teacher's Instructions}

These will be used to tell us what we are doing next on a regular basis :

\begin{enumerate}
\item \xpe{请打开书三页}{Please open your book 3 page.}
\item \xpe{我们学生词}{We learn new words.}
\item \xpe{请跟我读}{Please follow me reading aloud.}
\end{enumerate}

\subsection{Textbook Unit 1 / Lesson 1--3}

From the dialogues, we can form some standard phrases with the new vocabulary from lesson 1 surrounding basic introductions

\begin{enumerate}
\item \xpe{请问你叫什么}{Please to ask you be called what?}
\item \xpe{你叫什么名字}{You be called what name?}
\item \xpe{我认识他}{I to know him}
\item \xpe{很高兴认识你}{Very glad to know you.}
\item \xpe{认识你很高兴}{To know you very glad.}
\item \xpe{小姐}{Miss / Young girl / (prostitute, slang)}
\item \xpe{你姓什么}{You surname what?}
\item \xpe{}{}
\item \xpe{}{}
\item \xpe{}{}
\end{enumerate}

\subsection{Making Questions}

A good strategy is to work backwards from an answer, and then substitute the noun/pronoun of the answer for a question word.   e.g.   To work out how to ask for a name, start from the construction of the answer.  Thus ``he is called Mark'' transforms to ``he is called What?''.

In Mandarin :

\begin{question}
  他叫什么?
  \begin{tasks}(1)
  \task 他叫马克 (Ma-ke Mark)
  \end{tasks}
\end{question}

\begin{question}
  他姓什么?
  \begin{tasks}(1)
  \task 他姓Jones
  \end{tasks}
\end{question}

\subsection{Study Pattern}

When studying the text book dialogues :

\begin{enumerate}
\item Read aloud and pronounce the pinyin/characters.
\item Translate the meaning piecewise.
\item Synthesise new statements from the learned language.
\end{enumerate}

\subsection{Unit 1 / Lesson 2}

New particle word to indicate possessive like the apostrophe--'s' in English : 的 \xpinyin*{的} and handily the same as in French (de).

For example.  Mark's book.  Mark 的书.  Marke de shu1.

New word for a group of several people, as in an audience being addressed fairly formally and for the first time by a speaker (to gain attention) : 大家 (da4jia1) \xpinyin*{大家} - everybody.

\subsection{Pedagogical Points about Textbook}

The textbook asks negative questions by giving 3 statements, only 2 of which are true.  This is not helpful, so our tutor will indicate the errors in advance so we are not mislead.

\subsection{Character Study}

A new dictation sheet (DC1 Unit1) is provided with the followin gcharacters to learn.


\subsection{Homework}

\subsubsection{Dictation}

DC1 Unit 1 vocabulary.

\subsubsection{Oral}

Prepare to introduce yourself to others (apt).

\subsubsection{Written Work}

To be handed in and checked.  Can be received in arrears.

\subsubsection{Class Times/Rooms}

Tuesday 9-11 in 301.  Thursday 5-7 in 301.

\subsection{IP/Copyright}

In respect of the content, Dr. Wendy Che, and to an extent yet to be determined, her employer, the University of Oxford Language Center.

In respect of the write up and typesetting, Dr. Adam Vercingetorix Stephen.


\end{CJK} 
\end{document}
