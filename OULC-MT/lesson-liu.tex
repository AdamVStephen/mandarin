\documentclass{article}
% Margin control
\usepackage[a4paper, total={7in, 11in}]{geometry}
% For Chinese support, recommended by Chou : augmented with xpinyin by sobenlee@gmail.com
\usepackage{CJKutf8}
% For xpinyin
\usepackage{xpinyin}
\usepackage[T1]{fontenc}
\usepackage{lmodern}
% From Chou
\usepackage{ucs} 
\usepackage[encapsulated]{CJK}
% Font choices
% bsmi limited - lacks certain characters
%\newcommand{\myfont}{bsmi} % or {stheiti}, etc
% gbsn recommended from sobenlee@gmail.com in xpinyin documentation
\newcommand{\myfont}{gbsn} % or {stheiti}, etc
% For Exam/Question support taken from en.wikibooks.org/wiki/LaTeX/List\_Structures
\usepackage{tasks}
\usepackage{exsheets}
\SetupExSheets[question]{type=exam}
% Personal macros
\newcommand{\cvoc}[1]{#1 & \xpinyin*{#1} }
\newcommand{\cvct}[3]{#1 & \xpinyin*{#1} & \pinyin{#2} & #3 \\ \hline}
\newcommand{\cvctp}[4]{#1 & \xpinyin*{#1} & \pinyin{#2} & #3 & #4 \\ \hline}
%\newcommand{\cvctp}[4]{#1 & \xpinyin*{#1} & \pinyin{#2} & #3 & #4 \\ \hline}
% xpe = xpinyin : english
\newcommand{\xpe}[2]{\xpinyin*{#1} : #2}

\begin{document} 
\begin{CJK}{UTF8}{\myfont} 
\section{Lesson 六}

Thursday 22/11/2018

\subsection{Homework Checks}

\subsubsection{Dictation Practice}

From week five, we should know (be able to read and write) the characters :

Dictation exercises as given, with original and then correct answers.

\begin{question}
  \pinyin{ni3 baba zuo4 shen me gong1 zuo4}
  \xpinyin*{你爸爸做什么工作}
  \begin{tasks}(2)
    \task你爸爸X什么工X
    \task你爸爸做什么工作
  \end{tasks}
\end{question}

i.e. 5/7 correct.  Note the two forms for zuo4 (care - is it zu4o and zuo4?) 做 and 作.

\begin{question}
  \pinyin{wo3 didi shi xue2sheng}
  \xpinyin*{我弟弟是学生}
  \begin{tasks}(2)
    \task 我姐姐是生学
    \task 我弟弟是学生
  \end{tasks}
\end{question}

4/5 I confused 姐姐 for 弟弟.

\begin{question}
  \pinyin{ta1 mama zhu zai yi1yuan}
  \xpinyin*{他妈妈住在医院}

  \begin{tasks}(2)
    \task他妈妈X在医X
    \task他妈妈住在医院
  \end{tasks}
\end{question}

4/6 as did not remember 主 and had not properly learned 院

Overall score for dictation : 16 /17 (94.1\%).

\section{Character Review}

I now know approximately 50 characters from the main sheets and others picked up.

\subsection{Unit 4 : He is tall }

Introducing more discussion of people (general) and their occupations/characteristics.

New question word for discussing this topic is : who = 谁 pronounced shei2 in Northern accents, or shui2 in a southern accent.  Whereas the word 'who' is only used at the beginning of a construction in English, it can appear anywhere in a Chinese sentence as a general pronounce.   See the weblearn slides for some nice examples.   E.g.  

TODO : type up the weblearn slides and extract the main grammatical points.

A common topic during introductions may be to discuss people's age.   The measure words for number of years differ according to magnitude and so for young children (up to 10) one may ask '他几岁' but this would be impertinent (indicating childishness) to an adult.  The safer general construction is 'duo da sui, ni duo da, or simply duo da?' In passing, note that for an approximate age, one can form 'left-right = approximate' with 'zuo3you4'.   When asserting an age, one can omit the verb 是.   However, to contradict (he is {\it not} 28) for emphasis, the verb is strictly required. 我 不 是 28.  Note also in passing the oddity that characters which form compound words in written Chinese are not separated by spaces, but rather the full sentence runs together and must be parsed without the aid of whitespace.   Finally, the most compact form can be used in which both the verb (to be) and the noun (years) can be omitted and it is understood to say simply 我35. (I 35 = I am 35 = I am 35 years old.).

  
Useful vocabulary :

    \begin{tabular}{|l|l|l|l|l|} \hline
      \cvctp{}{}{}{}
      \cvctp{}{}{}{}
      \cvctp{}{}{}{}
      \cvctp{}{}{}{}
      \cvctp{}{}{}{}
      \cvctp{}{}{}{}
      \cvctp{}{}{}{}
      \cvctp{}{}{}{}
      \cvctp{}{}{}{}
      \cvctp{}{}{}{}
      \cvctp{}{}{}{}
      \cvctp{}{}{}{}
      \cvctp{}{}{}{}
      \cvctp{}{}{}{}
      \cvctp{}{}{}{}
      \cvctp{}{}{}{}
      \cvctp{}{}{}{}
      \cvctp{}{}{}{}
      
    \end{tabular}

    Phrases in passing

    \begin{enumerate}
    \item \xpe{你作什么工作?}{You do what work?}
    \item \xpe{我在牛津工作}{I in Oxford work. (note place before verb)}
    \item \xpe{我们都是学生}{We all are students.}
    \item \xpe{你家即口人}{Your family how many (total person measure word) people?}
    \item \xpe{我有三口人}{My family has 3 (total person measure word) people.}
    \item \xpe{我有两个积极}{I have two (measure word general) elder sisters.}
    \item \xpe{我妹妹是学生。我姐姐是老师}{My younger sister is a student.  My elder sister is a teacher.}
    \item \xpe{}{xiong di = all brothers (to do look up characters)}
    \item \xpe{}{jia mei = all sisters (to do look up characters) }
    \item \xpe{}{}
    \end{enumerate}


      
      \subsection{Grammar, Adverbs}

      Recall - subject, adverb, verb, object.  Adverbs come before the verb always.   The order will in general be whatever is correct to imply the intended meaning.  e.g. If combining 不, 都 then we can have either 不都 meaning not all, or 都不 meaning all not, i.e. none.   THe adverb 也 is special and almost always precedes all other adverbs.

      The structural particle 的 is used to indicate the posessive and can thus be used as an apostrophical suffix to replcae similar usage in English.   Thus Adam 的书 is Adam's book.  For entities with close relationships it is permissible to omit the particle which could otherwise yield unwieldy repetition.   Consider brother's uncle's home's location which would be 弟弟的爸爸的家的元.  Note all the ``de''s.   So one can talk about 我的书 but simply 我妹妹.

      Measure words and why two has two forms.   When referring to a pair of younger sisters, one would not use the ordinal/counting form of the second positive integer 二 but instead the variant 两.   When referring to numbers of nouns a linking {\it measure word} is required between the number and the noun.  The most generic (best guess) is 个 and so one would refer to 两个弟弟 or 三个妹妹 but there is a specific measure word for all the mouths in the family (i.e. the total number of family members) which is 口.

      The conjunction and (和)is not used to join two sentences.   Just split them.


\subsection{New Words}

Taken from the vocabulary in the text book, unit 1, supplemented with some of the other words introduced in the lesson.

    \begin{tabular}{|l|l|l|l|l|} \hline
      \cvctp{您}{nin}{adv}{You (polite)} 
    \end{tabular}

    \subsection{Culture and Usage}
\subsection{Character Study}

A new dictation sheet (DC1 Unit1) is provided with the followin gcharacters to learn.

\subsection{Homework}

\subsubsection{Dictation}

DC1 Unit 2 vocabulary.

\subsubsection{Oral}

Prepare to introduce your family to others.

\subsubsection{Written Work}

To be handed in and checked.  New vocab and grammar sheets to be completed.

\subsubsection{Class Times/Rooms}

Tuesday 9-11 in 301.  Thursday 5-7 in 301.

\subsection{IP/Copyright}

In respect of the content, Dr. Wendy Che, and to an extent yet to be determined, her employer, the University of Oxford Language Center.

In respect of the write up and typesetting, Dr. Adam Vercingetorix Stephen.

\end{CJK} 
\end{document}
